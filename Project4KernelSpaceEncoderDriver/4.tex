\documentclass{article}
\usepackage[T1]{fontenc}
\usepackage[utf8]{inputenc}
\usepackage[a4paper, total={6in, 8in}]{geometry}
%\usepackage[icelandic]{babel}
\usepackage{graphicx} %package to manage images

\usepackage{hyperref}
\usepackage{siunitx}
\usepackage{tabularx}

\usepackage{xcolor}
\usepackage{listings}

\colorlet{mygray}{black!30}
\colorlet{mygreen}{green!60!blue}
\colorlet{mymauve}{red!60!blue}

\lstset{
  backgroundcolor=\color{gray!10},  
  basicstyle=\ttfamily,
  columns=fullflexible,
  breakatwhitespace=false,      
  breaklines=true,                
  captionpos=b,                    
  commentstyle=\color{mygreen}, 
  extendedchars=true,              
  frame=single,                   
  keepspaces=true,             
  keywordstyle=\color{blue},      
  language=c++,                 
  numbers=none,                
  numbersep=5pt,                   
  numberstyle=\tiny\color{blue}, 
  rulecolor=\color{mygray},        
  showspaces=false,               
  showtabs=false,                 
  stepnumber=5,                  
  stringstyle=\color{mymauve},    
  tabsize=3,                                     
  title=\lstname 
}

\title{Embedded Group Project\\ \large Project 4 - Kernel Space Encoder}

\author{Steinarr Hrafn Höskuldsson\\
Arnþór Gíslason\\
Andrew Madden\\
\\
Reykjavik University}
\date{October 2022}


\newcommand{\mycomment}[1]{}
\newcommand{\timerinterval}{5ms }

\begin{document}
\maketitle
 % how to comment, input image and code
\mycomment{
\begin{figure}[h]
    \centering
    \includegraphics[width=0.75\textwidth]{Project3ControllerStateMachine/out/Project3ControllerStateMachine/docs/uml/uml.png}
    \caption{UML}
    \label{fig:UML}
\end{figure}

\lstinputlisting[caption=Defining 'ColorMatch' state, label={lst:colormatch}, language=Python, firstline=44, lastline=52]{LAB3/Basic.py}

}

\section{Part 1}
The header pins were soldered onto the Raspberry Pi Zero W2 and a voltmeter used to check that the pin orientation was correct.
\section{Part 2}
The four different methods for interfacing with the GPIO pins were tested.
\subsection{Test Setup}
A function generator ( WHAT MODEL?) was hooked up to pin ????????? and set to output a square wave with amplitude 3 volts and frequency 1kHz. A Rhode\&Schwartz RBT2004 oscilloscope was used to probe the input and output pins on different channels. The oscilloscope was then set to measure the time difference between rising edges.

\subsection{Using sysfs from a shell script}

\subsection{Using sysfs from a C/C++ application}
\subsection{Using sysfs and poll() from a C/C++ application}
\subsection{A kernel module using interrupts}


\section{Part 3}

\section*{Appendix}
\appendix

\newpage
\section{Code}\label{appendix:code}

\lstinputlisting[caption=main program used to produce the step response]{Project2SpeedController/src/main.cpp}

\lstinputlisting[caption=timer\_msec.cpp]{Project1RotaryEncoder/src/encoder_simple.cpp}

\lstinputlisting[caption=main.cpp]{Project1RotaryEncoder/src/main.cpp}

\end{document}
